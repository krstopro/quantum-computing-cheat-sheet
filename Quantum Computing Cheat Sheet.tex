\documentclass[12pt,a4paper]{article}

\usepackage{amsmath}
\input{Qcircuit}
\allowdisplaybreaks

\newcommand{\CNOT}{\text{CNOT}}

\title{Quantum Computing Cheat Sheet}
\date{}

\begin{document}


\maketitle

\section{States}

\begin{equation*}
\ket{0} = \begin{bmatrix} 1 \\ 0 \end{bmatrix}
\quad
\ket{1} = \begin{bmatrix} 0 \\ 1 \end{bmatrix}
\end{equation*}

\begin{equation*}
\ket{+} = \frac{\ket{0} + \ket{1}}{\sqrt{2}} 
= \begin{bmatrix} \frac{1}{\sqrt{2}} \\ \frac{1}{\sqrt{2}} \end{bmatrix}
\quad
\ket{-} = \frac{\ket{0} - \ket{1}}{\sqrt{2}}
= \begin{bmatrix} \frac{1}{\sqrt{2}} \\ -\frac{1}{\sqrt{2}} \end{bmatrix}
\end{equation*}

\begin{equation*}
\ket{\psi} = \alpha \ket{0} + \beta \ket{1} = \begin{bmatrix} \alpha \\ \beta \end{bmatrix} ,
\quad
|\alpha|^2 + |\beta|^2 = 1
\end{equation*} \\

\section{Unitary Operators}

\begin{tabular}{r@{=}l}
\end{tabular}

\begin{alignat*}{3}
I &= \begin{bmatrix} 1 & 0 \\ 0 & 1 \end{bmatrix} \hspace{95pt}
\begin{aligned}
I \ket{0} &= \ket{0} \\
I \ket{1} &= \ket{1}
\end{aligned} &
\Qcircuit @C=1em @R=.7em {
	& \qw & \qw & \qw
} \\[12pt]
X &= \begin{bmatrix} 0 & 1 \\ 1 & 0 \end{bmatrix} \hspace{95pt}
\begin{aligned}
X \ket{0} &= \ket{1} \\
X \ket{1} &= \ket{0}
\end{aligned} &
\Qcircuit @C=1em @R=.7em {
	& \gate{X} & \qw
} \\[12pt]
Y &= \begin{bmatrix} 0 & -i \\ i & 0 \end{bmatrix} \hspace{88pt}
\begin{aligned}
Y \ket{0} &= i\ket{1} \\
Y \ket{1} &= -i\ket{0}
\end{aligned} &
\Qcircuit @C=1em @R=.7em {
	& \gate{Y} & \qw
} \\[12pt]
Z &= \begin{bmatrix} 1 & 0 \\ 0 & -1 \end{bmatrix} \hspace{86pt}
\begin{aligned}
Z \ket{0} &= \ket{0} \\
Z \ket{1} &= -\ket{1}
\end{aligned} &
\Qcircuit @C=1em @R=.7em {
	& \gate{Z} & \qw
} \\[12pt]
H &= \frac{1}{\sqrt{2}} \begin{bmatrix} 1 & 1 \\ 1 & -1 \end{bmatrix} \hspace{66pt}
\begin{aligned}
H \ket{0} &= \ket{+} \\
H \ket{1} &= \ket{-}
\end{aligned} &
\Qcircuit @C=1em @R=.7em {
	& \gate{H} & \qw
} \\[12pt]
S &= \begin{bmatrix} 1 & 0 \\ 0 & i \end{bmatrix} \hspace{96pt}
\begin{aligned}
S \ket{0} &= \ket{0} \\
S \ket{1} &= i\ket{1}
\end{aligned} &
\Qcircuit @C=1em @R=.7em {
	& \gate{S} & \qw
} \\[12pt]
T &= \begin{bmatrix} 1 & 0 \\ 0 & e^{i \pi / 4} \end{bmatrix} \hspace{80pt}
\begin{aligned}
T \ket{0} &= \ket{0} \\
T \ket{1} &= e^{i \pi / 4}\ket{1}
\end{aligned} &
\Qcircuit @C=1em @R=.7em {
	& \gate{T} & \qw
} \\[12pt]
\CNOT &= \begin{bmatrix} 1 & 0 & 0 & 0 \\ 
                                           0 & 1 & 0 & 0 \\
                                           0 & 0 & 0 & 1 \\
                                           0 & 0 & 1 & 0
                  \end{bmatrix} \hspace{62pt}
\begin{aligned}
\CNOT \ket{00} &= \ket{00} \\
\CNOT \ket{01} &= \ket{01} \\
\CNOT \ket{10} &= \ket{11} \\
\CNOT \ket{11} &= \ket{10}
\end{aligned} & \hspace{72pt}
\Qcircuit @C=1em @R=.7em {
	& \ctrl{2} & \qw \\
	& \\
	& \targ & \qw
}
\end{alignat*} \\

\section{Operator identities}

\begin{align*}
X^2 = Y^2 = Z^2 = H^2 = I
\end{align*}

\begin{equation*}
T^2 = S \quad S^2 = Z
\end{equation*}

\begin{align*}
XY = iZ \quad YX = -iZ \quad XZ = -iY \quad ZX = iY \quad YZ = -iZ \quad ZY = -iX
\end{align*}

\begin{align*}
HX = ZH \quad HZ = XH \\
SX = XZS \quad SZ = ZS
\end{align*}

\begin{align*}
HXH = Z \quad HYH = -Y \quad HZH \\
SXS = Y \quad SYS = X \quad SZS = Z
\end{align*}

\begin{equation*}
\Qcircuit @C=1em @R=.7em {
	& \gate{H} & \ctrl{2} & \gate{H} & \qw & & & \targ & \qw \\
	& & & & & \push{\rule{.12em}{0em}=\rule{.12em}{0em}} 
	& & & & & &
	\push{\rule{0.12em}{0em}
	(H \otimes H) \CNOT_{0,1} (H \otimes H) = \CNOT_{1,0}
	} \\
	& \gate{H} & \targ & \gate{H} & \qw & & & \ctrl{-2} & \qw
}
\end{equation*} \\

\begin{equation*}
\Qcircuit @C=1em @R=.7em {
	& \ctrl{2} & \gate{X} & \qw & & & \gate{X} & \ctrl{2} & \qw  \\
	& & & & \push{\rule{.12em}{0em}=\rule{.12em}{0em}} 
	& & & & & & &
	\push{\rule{0.12em}{0em}
	\CNOT_{0,1}(X \otimes I) = (X \otimes X) \CNOT_{0,1}
	} \\
	& \targ & \qw & \qw & & & \gate{X} & \targ & \qw
}
\end{equation*} \\

\begin{equation*}
\Qcircuit @C=1em @R=.7em {
	& \ctrl{2} & \qw & \qw & & & \qw & \ctrl{2} & \qw  \\
	& & & & \push{\rule{.12em}{0em}=\rule{.12em}{0em}} 
	& & & & & & &
	\push{\rule{0.12em}{0em}
	\CNOT_{0,1}(I \otimes X) = (I \otimes X) \CNOT_{0,1}
	} \\
	& \targ & \gate{X} & \qw & & & \gate{X} & \targ & \qw
}
\end{equation*} \\

\begin{equation*}
\Qcircuit @C=1em @R=.7em {
	& \ctrl{2} & \gate{Z} & \qw & & & \gate{Z} & \ctrl{2} & \qw  \\
	& & & & \push{\rule{.12em}{0em}=\rule{.12em}{0em}} 
	& & & & & & &
	\push{\rule{0.12em}{0em}
	\CNOT_{0,1}(Z \otimes I) = (Z \otimes I) \CNOT_{0,1}
	} \\
	& \targ & \qw & \qw & & & \qw & \targ & \qw
}
\end{equation*} \\

\begin{equation*}
\Qcircuit @C=1em @R=.7em {
	& \ctrl{2} & \qw & \qw & & & \gate{Z} & \ctrl{2} & \qw  \\
	& & & & \push{\rule{.12em}{0em}=\rule{.12em}{0em}} 
	& & & & & & &
	\push{\rule{0.12em}{0em}
	\CNOT_{0,1}(I \otimes Z) = (Z \otimes Z) \CNOT_{0,1}
	} \\
	& \targ & \gate{Z} & \qw & & & \gate{Z} & \targ & \qw
}
\end{equation*} \\

\end{document}